\chapter{ERP como soluci\'on adecuada}
%
La necesidad del negocio se puede resumir en los siguientes puntos:%
%
\begin{itemize}
\item Gesti\'on y soporte de los procesos para optimizaci\'on constante.
\item Control y seguimiento para permitir toma de desiciones.
\item Interconexi\'on e interoperabilidad entre las sucursales.
\end{itemize}
%
Teniendo en cuenta esto, la soluci\'on m\'as acertada es la implementaci\'on de un ERP, debido a las caracter\'isticas mencionadas anteriormente.%
\\%
\\%
Un sistema ERP permite gesti\'on unificada y descentralizada, permitiendo as\'i alcance a las sucursales sin traumatismos, permiti\'endoles ejercer como unidades independientes con un fin y una directriz en com\'un, con interacci\'on de forma asincr\'onica y en tiempo real, facilitando la gesti\'on y revisi\'on de los procesos.%
