\chapter{ERP como soluci\'on adecuada}
%
La necesidad del negocio se puede resumir en los siguientes puntos:%
%
\begin{itemize}
\item Gesti\'on y soporte de los procesos para optimizaci\'on constante.
\item Control y seguimiento para permitir toma de desiciones.
\item Interconexi\'on e interoperabilidad entre las sucursales.
\end{itemize}
%
Teniendo en cuenta esto, la soluci\'on m\'as acertada es la implementaci\'on de un ERP, debido a las caracter\'isticas mencionadas anteriormente.%
\\%
\\%
Un sistema ERP permite gesti\'on unificada y descentralizada, permitiendo as\'i alcance a las sucursales sin traumatismos, permiti\'endoles ejercer como unidades independientes con un fin y una directriz en com\'un, con interacci\'on de forma asincr\'onica y en tiempo real, facilitando la gesti\'on y revisi\'on de los procesos.%

\chapter{Marco Referencial}
%
Es un  software que   automatiza e integra  los procesos de la organizaci\'on  as\'i como la  producci\'on y distribuci\'on.
%
\\%
\\%
Estos sistemas  unen y sincronizan todas las operaciones de la organizaci\'on tambi\'en  permitiendo tener una conexi\'on entre la empresa  con sus clientes y proveedores.
\\%
\\%
Permitiendo una interfaz con el usuario para ejecutar las transacciones de la empresa y bases de datos centralizada para almacenar toda la informaci\'on. Se consideran como  la  integraci\'on de los diferentes sistemas de informaci\'on en todas las \'areas de las empresas.
\\%
\\%
Nuestro ERP nos permite:
\\%
\begin{itemize}
	\item Control,Gesti\'on y planeaci\'on de recursos.
	\item Planeaci\'on de productos.
	\item Gesti\'on de costos.
	\item Manejos de inventarios.
	\item F\'acil acceso a proveedores.
	\item Manejo de insumo.
\end{itemize}
%
El ERP consta de un flujo constante de informaci\'on entre las \'areas de la organizaci\'on las cuales tendr\'an un trabajo individual pero al mismo tiempo una coordinaci\'on e integraci\'on de los procesos que conforman el ERP.
\\%
Con la automatizaci\'on de nuestra empresa por medio del ERP buscaremos:
%
\begin{itemize}
	\item Mejorar el flujo de procesos. 
	\item	Obtener un mejor an\'alisis y respuestas.
	\item Disminuir costos de producci\'on y gesti\'on de inventario.
	\item Facilitar el servicio al cliente.
	\item Tener un control de insumos necesarios.
\end{itemize}
%
Despu\'es de un determinado tiempo de automatizar nuestra organizaci\'on por medio del ERP obtendremos resultados y justificaremos nuestros costos de inversi\'on obteniendo como resultados  mejora de procesos y tiempos de entregas reduciendo , as\'i mismo tiempo de espera  aumentando as\'i las ventas por un mejor servicio a nuestros clientes y obtendremos una mayor competitividad  en el mercado por nuestros servicios  internamente nuestra Organizaci\'on obtendr\'a un incremento notorio de precisi\'on de inventario y mejor gesti\'on de insumos.
\\%
\\%
Al ser un medio tecnol\'ogico se busca tener el f\'acil acceso y ejecuci\'on de los usuarios asi mismo los empleados tendr\'an una capacitaci\'on de c\'omo implementar esta nueva tecnolog\'ia ya que se le asignara responsabilidades m\'as grandes en la organizaci\'on. El ERP puede ser muy costoso al principio de su inversi\'on pero al manejar el f\'acil flujo de informaci\'on en los procesos de cada \'area de la empresa ser\'a una inversi\'on f\'acil de recuperar y de mejor proyecci\'on para su organizaci\'on.%