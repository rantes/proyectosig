\chapter{Planeaci\'on}
\section{Planeaci\'on Organizacional}
\subsection{Misi\'on}
Trabajamos duro en familia para crear productos especiales e innovadores, comprometidos en acompa\~nar momentos \'unicos, generando valor para todos los integrantes.%
%
\subsection{Visi\'on}
Ser reconocidos como una panader\'ia pasteler\'ia innovadora, especial y \'unica, en b\'usqueda de lo mejor.%
%
\subsection{Valores}
\begin{itemize}
\item \emph{Honestidad}: Somos coherentes entre lo que decimos y lo que hacemos.%
\item \emph{Familiaridad}: Hornitos, dulce hogar.%
\item \emph{Compromiso}: Somos responsables de nuestros actos.%
\item \emph{Trabajo Duro}: Damos siempre la milla extra.%
\item \emph{Calidad}: Hacemos las cosas bien desde la primera vez.%
\end{itemize}
%
\section{Plan estr\'ategico de ventas}
\begin{figure}[htbp]
%centering es para centrar la imagen
	\centering
%aca es donde se incluye la imagen, se da el ancho(width), \textwidth significa que con repescto al tamano del
%texto y luego la ruta, relativa siempre es decir, a partir de donde se esta, como images esta ahi
%dentro, solo se usa desde images y ojala nada de espacios en el nombre de la imagen
		\includegraphics[width=0.60\textwidth]{images/Dibujo.jpg}
%el caption es para el texto que aparece debajo de la imagen
	\caption{Plan estrategico}
%label es para darle una referencia, por ejemplo si uno dice "como se puede ver en la imagen a1"
	\label{fig:Plan estrategico}
\end{figure}%
Hornitos S.A. Ha desarrollado una campa\~na la cual se dividir\'a en 3 partes las cuales buscara tener  por medio de los clientes una preferencia en nuestros productos e incentivar a nuestros trabajadores a tener mayor entrega con la organizaci\'on estar\'iamos hablando de tres proyectos los cuales buscaran un mejor \'ambito para la organizaci\'on.%
\\%
	
	\begin{itemize}
		\item Incentivos empleados.
		\item Degustaciones.
		\item bonos y rifas.
	\end{itemize}
	
\subsection{Incentivos empleados:}Cada sucursal tendr\'a un jefe de \'area el cual estar\'a a cargo las ventas con su  respectivo equipo de trabajo.  Tendr\'an una comisi\'on base de ventas pero con un incentivo adicional para los trabajadores y el jefe de la sucursal si logra sobre pasar ventas mensuales a 12 millones.%
\subsection{Degustaciones:} Con el fin de promocionar nuestro producto y generar una mayor demanda de ventas  efectuaremos degustaciones en diferentes centros cerca de nuestras sucursales para generar m\'as confianza entre los clientes y as\'i ser preferidos para ellos.
%
\subsection{Bonos y Rifas:} En nuestra  organizaci\'on lo m\'as importante es contener nuestros clientes y con este fin pensando en ello generaremos rifas en fechas especiales y tambi\'en otorgaremos bonos de descuentos para que lo haga efectivo en nuestras sucursales solo para la fidelidad de nuestros clientes 
%\\%
%\\%
%
%
%\newpage%
%\section{Reportes}
%\begin{figure}[htbp]
%%centering es para centrar la imagen
%	\centering
%%aca es donde se incluye la imagen, se da el ancho(width), \textwidth significa que con repescto al tamano del
%%texto y luego la ruta, relativa siempre es decir, a partir de donde se esta, como images esta ahi
%%dentro, solo se usa desde images y ojala nada de espacios en el nombre de la imagen
%		\includegraphics[width=0.60\textwidth]{images/REPORTEDEINSUMO.jpg}
%%el caption es para el texto que aparece debajo de la imagen
%	\caption{Reporte de insumo}
%%label es para darle una referencia, por ejemplo si uno dice "como se puede ver en la imagen a1"
%	\label{fig:reportedeinsumo}
%\end{figure}%
%\\%
%\\%
%\\%
%\\%
%\begin{figure}[htbp]
%%centering es para centrar la imagen
%	\centering
%%aca es donde se incluye la imagen, se da el ancho(width), \textwidth significa que con repescto al tamano del
%%texto y luego la ruta, relativa siempre es decir, a partir de donde se esta, como images esta ahi
%%dentro, solo se usa desde images y ojala nada de espacios en el nombre de la imagen
%		\includegraphics[width=0.60\textwidth]{images/REPORTEDEVENTASGENERAL.jpg}
%%el caption es para el texto que aparece debajo de la imagen
%	\caption{Reporte de ventas general}
%%label es para darle una referencia, por ejemplo si uno dice "como se puede ver en la imagen a1"
%	\label{fig:Reporte de ventas general}
%\end{figure}%
%\\%
%\\%
%\\%
%\\%
%\begin{figure}[htbp]
%%centering es para centrar la imagen
%	\centering
%%aca es donde se incluye la imagen, se da el ancho(width), \textwidth significa que con repescto al tamano del
%%texto y luego la ruta, relativa siempre es decir, a partir de donde se esta, como images esta ahi
%%dentro, solo se usa desde images y ojala nada de espacios en el nombre de la imagen
%		\includegraphics[width=0.60\textwidth]{images/REPORTEDEVENTASDELASUCURSAL.jpg}
%%el caption es para el texto que aparece debajo de la imagen
%	\caption{Reporte de ventas de una sucursal}
%%label es para darle una referencia, por ejemplo si uno dice "como se puede ver en la imagen a1"
%	\label{fig:Reporte de ventas de una sucursal}
%\end{figure}%
%\newpage%
%\section{Dise\~no de interfaces}
%El ingreso al aplicativo est\'a marcado por el inicio de sesi\'on donde se autoriza el ingreso al aplicativo, teniendo en cuenta el nivel de permisos, dado que la informaci\'on all\'i almacenada no es de uso p\'ublico.
%\\%
%Inicialmente esta ventana:
%%\begin{figure}[htbp]
%%%centering es para centrar la imagen
%%	\centering
%%%aca es donde se incluye la imagen, se da el ancho(width), \textwidth significa que con repescto al tamano del
%%%texto y luego la ruta, relativa siempre es decir, a partir de donde se esta, como images esta ahi
%%%dentro, solo se usa desde images y ojala nada de espacios en el nombre de la imagen
%%		\includegraphics[width=0.60\textwidth]{images/Iniciosecion.jpg}
%%%el caption es para el texto que aparece debajo de la imagen
%%	\caption{Inicio de secion}
%%%label es para darle una referencia, por ejemplo si uno dice "como se puede ver en la imagen a1"
%%	\label{fig:Inicio de secion}
%%\end{figure}%
%\\%
%En el que evaluara que el Usuario y la Contrase\~na sean las correctas, a partir de esto el programa se encargar\'a que opciones del aplicativo le permita manipular.
%\\%
%\\%
%Seguido a esto despu\'es de validar las contrase\~nas ofrecer\'a el siguiente men\'u:
%\\%
%\\%
%%
%%\begin{figure}[htbp]
%%%centering es para centrar la imagen
%%	\centering
%%%aca es donde se incluye la imagen, se da el ancho(width), \textwidth significa que con repescto al tamano del
%%%texto y luego la ruta, relativa siempre es decir, a partir de donde se esta, como images esta ahi
%%%dentro, solo se usa desde images y ojala nada de espacios en el nombre de la imagen
%%		\includegraphics[width=0.60\textwidth]{images/Administrador.jpg}
%%%el caption es para el texto que aparece debajo de la imagen
%%	\caption{Ventana administrador}
%%%label es para darle una referencia, por ejemplo si uno dice "como se puede ver en la imagen a1"
%%	\label{fig:Ventana administrador}
%%\end{figure}%
%\\%
%\\%
%En este despliegue de men\'u se seleccionar\'a el procedimiento solicitado, relacionado a los empleados.
%\\%
%\\%
%%\begin{figure}[htbp]
%%%centering es para centrar la imagen
%%	\centering
%%%aca es donde se incluye la imagen, se da el ancho(width), \textwidth significa que con repescto al tamano del
%%%texto y luego la ruta, relativa siempre es decir, a partir de donde se esta, como images esta ahi
%%%dentro, solo se usa desde images y ojala nada de espacios en el nombre de la imagen
%%		\includegraphics[width=0.60\textwidth]{images/Segundoadministrador.jpg}
%%%el caption es para el texto que aparece debajo de la imagen
%%	\caption{Ventana administrador}
%%%label es para darle una referencia, por ejemplo si uno dice "como se puede ver en la imagen a1"
%%	\label{fig:Ventana administrador}
%%\end{figure}%
%\\%
%\\%
%En este despliegue de men\'u se seleccionara el procedimiento solicitado referente al inventario.
%\\%
%\\%
%%\begin{figure}[htbp]
%%%centering es para centrar la imagen
%%	\centering
%%%aca es donde se incluye la imagen, se da el ancho(width), \textwidth significa que con repescto al tamano del
%%%texto y luego la ruta, relativa siempre es decir, a partir de donde se esta, como images esta ahi
%%%dentro, solo se usa desde images y ojala nada de espacios en el nombre de la imagen
%%		\includegraphics[width=0.60\textwidth]{images/Facturas.jpg}
%%%el caption es para el texto que aparece debajo de la imagen
%%	\caption{Ventana factura}
%%%label es para darle una referencia, por ejemplo si uno dice "como se puede ver en la imagen a1"
%%	\label{fig:Ventana factura}
%%\end{figure}%
%\\%
%\\%
%Este men\'u se prioriza para realizar las facturas, con el fin de solicitar informaci\'on pertinente que sea necesaria.
%\\%
%\\%
%\mbox{}
%\newpage%
%%
%\mbox{}
%\newpage%
%%
%\section{Recursos}
%%
%\mbox{}
%\newpage%
%\mbox{}
%\newpage%
%\mbox{}
%\newpage%
%\mbox{}
%\newpage%
%\section{Presupuesto}
%%
%\mbox{}
%\newpage%
%%
%\mbox{}
%\newpage%
%%
%\section{Cronograma}
%%
%\mbox{}
%\newpage%
%\mbox{}
%\newpage%
%\mbox{}
%\newpage%
%\mbox{}
%\newpage%