\chapter*{introducci\'on}
El motor fundamental de toda organizaci\'on es la generaci\'on de ingresos, y entre m\'as se 
generen, mejor ser\'an los resultados y puede mantenerse en el mercado. El factor crucial son las ventas efectivas que pueda realizar y el n\'umero de clientes que pueda adquirir y fidelizar. Para tal prop\'osito, se crean planes de incentivaciones a ventas y fidelizaci\'on de clientes.%
\\%
Existen muchas maneras de lograr estos planes pero en ultima instancia, el manejo de los datos de los mismos, puede tornarse engorroso y mas, trat\'andose de altos vol\'umenes de transacciones de informaci\'on.%
\\%
\\%
En las campa\~nas de fidelizaci\'on, tambi\'en se busca que esos clientes crezcan de la mano de la organizaci\'on, de forma tal que, el sistema sea simbi\'otico; no solo mantener los clientes o conseguir nuevos, sino que se les motive a comprar m\'as.%
\\%
\\%
En este plan de trabajo, se har\'a enfoque hacia las campa\~nas a clientes, para poder mostrar la importancia de tener una buena herramienta que soporte un sistema de informaci\'on.%
\\%
\\%
La finalidad de implementar un nuevo sistema de Gesti\'on Empresarial \(ERP\). El volumen de ventas que registran mes a mes es bastante alto y sus clientes reportan hasta compras por ordenes de \$400.000.000 al mes por cliente. Son una empresa comprometida con sus empleados y con sus clientes y siempre est\'a buscando planes que le permitan mejorar tanto la calidad de vida de los mismos como el nombre de la empresa.%
\\%
\\%
El Grupo Brinsa S.A. ha ideado 2 campa\~nas, una enfocada a la fuerza de ventas que ha denominado "BrinsaClub" y otra denominada "BrinsaMan\'ia", enfocada en sus clientes mayoristas. El objetivo de este desarrollo es construir una herramienta que brinde soporte a la campa\~na de BrinsaMan\'ia, permitiendo que la organizaci\'on pueda llevar a cabo sus cometidos trazados para este plan.%