\chapter*{Introducci\'on}
Las redes sociales en los \'ultimos 3 a\~nos, se han convertido en un punto convergente donde las personas interactuan mas fuertemente que en la vida real, terminando hasta en llevar su vida basada en las redes sociales.%
\\%
\\%
La mayor\'ia de las organizaciones est\'a siendo conciente de este efecto y muchos han decidido invader las redes sociales para acercarse m\'as a sus clientes.%
\\%
\\%
En este acercamiento, surge una nueva profesi\'on: el manejo de comunidades; teniendo simplemente personas que administran un contenido de acuerdo a las desiciones de las directivas de la organizaci\'on.%
\\%
\\%
La mayor\'ia de los profesionales dedicados a esta labor son comunicadores sociales debido a su dominio en cuanto a creaci\'on de contenidos pero, no son diestros con las apliaciones ofim\'aticas o con labores estad\'isticas. Estas acciones concluyen en una menospreciaci\'on de la labor y del personaje encargado.%
\\%
\\%
La idea propuesta en este escrito pretende ofrecer una opci\'on realmente profesional a este tema, permitiendo velorizar el proceso de la gesti\'on de relaci\'on con los clientes a nivel de las redes sociales.%