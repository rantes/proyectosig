\chapter{An\'alisis del sistema}
\section{Identificaci\'on de problemas y oportunidades b\'asicas}
En el estudio del sistema, se encontraron los siguientes problemas:%
\\%
\begin{itemize}
	\item El sistema debe ser flexible y parametrizable.%
	\item El volumen del flujo de datos es muy grande.%
	
	
\end{itemize}%
%
\begin{figure}[htbp]
%centering es para centrar la imagen
	\centering
%aca es donde se incluye la imagen, se da el ancho(width), \textwidth significa que con repescto al tamano del
%texto y luego la ruta, relativa siempre es decir, a partir de donde se esta, como images esta ahi
%dentro, solo se usa desde images y ojala nada de espacios en el nombre de la imagen
		\includegraphics[width=0.60\textwidth]{images/Eficienciadelasucursal.jpg}
%el caption es para el texto que aparece debajo de la imagen
	\caption{Eficiencias de las sucursales}
%label es para darle una referencia, por ejemplo si uno dice "como se puede ver en la imagen a1"
	\label{fig:eficienciasucursales}
\end{figure}%
%
\section{Evaluaci\'on del beneficio del proyecto}
Lo beneficios de el sistema de informaci\'on que se va a implementar nos permitir\'a consultar, observar y verificar la informaci\'on de la compa\~n\'ia en tiempo real y efectiva, de tal forma que estos procesos puedan ser flexibles y f\'acil de manejar para el usuario, esto nos lleva a que se Incrementara  el entendimiento del uso de las Tecnolog\'ias de Informaci\'on en la organizaci\'on y posteriormente  incrementar las ventas; para llevar esto a cabo, es necesario ser el primero en proporcionar informaci\'on a clientes potenciales . y mantener una relaci\'on con ellos a trav\'es de informaci\'on permanente.%
%
\subsection{?`Vale la pena trabajar en este proyecto?}
Hoy en d\'ia se esta tomando conciencia de la importancia dentro de las organizaciones de manejar los ambientes de negocio implementando un sistema de informaci\'on, lo fundamental para una compa\~n\'ia es un mercado globalizado a nivel competitivo, que tanto la estrategia de negocios y estrategias tecnol\'ogicas est\'en bajo un mismo esquema.%
\\%
\\%
Por lo anterior vale la pena trabajar en este proyecto ya que se ve reflejado la fusi\'on de estas dos estrategias, por una parte se esta trabajando los procesos del negocio y por el otro, se menejara un sistema que ayudar\'a a que todos estos procesos se realicen en forma automatizada, con el fin de alcanzar las metas y objetivos de la empresa.%
%
\subsection{?`Solucionar\'a los problemas?}
Uno de los principales inconvenientes de la organizaci\'on es que tiene gran cantidad de quejas y opiniones de los clientes esto lleva a que no se tenga en cuenta las opiniones de los clientes por ser una fuente de datos muy grande por eso se realizara este sistema, ya que el CRM es un sistema muy efenctivo en procesos frente al cliente por eso se implemetara, y esto sera para generar una mejor atecion al cliente, trato de todos los datos generando reportes y estadisticas.%
%
\subsection{Agregar valor al proceso}
Se agrega valor en los siguientes aspectos:%
\begin{enumerate}
	\item Proporcionando informacion adicional para la alta gerencia la cual ayuda para la toma de decisiones.
	\item Identificando inconvenientes que si resuelven y mejora la competencia y el incremento de las venta.
	\item Identificando las oportunidades de mejora en todas las \'areas.
\end{enumerate}%
%
\section{Dominio del problema}
El sistema que es implementado de forma sencilla no tan complejo ya que son pocos procesos que intervienen; siendo as\'i un sistema con 3 actores directos que son los comunity managers, los operadores del contact center y la organizaci\'on (due\~no de las campa\~nas del social media) a quien se le va a gestionar sus redes sosciales, en el que puede llegar a derivarse en altos niveles directivos y analistas de mercados. Actores indirectos son la otra faceta de este sistema y est\'a con formado por los clientes de la organizacion del social media, es decir, por dar un ejemplo: McDonalds necesita monitorear su impacto con sus clientes en las redes sociales; McDonalds se convierte en cliente del sistema y sus clientes van a hacer el objetivo de monitorizaci\'on.%
%
\section{An\'alisis de los procesos del negocio}
Los principales procesos est\'an identificados de la siguiente manera:%
	\begin{itemize}
		\item Campa\~nas.
		\item Satisfacci\'on al cliente.
	\end{itemize}
\subsection{Campa\~nas:}
Se realizan campa\~nas para las diferentes tareas en la organizaci\'on, tales como comunicados, informaci\'on, aperturas, promoci\'on de productos y promoci\'on de empleos entre otras.%
%
\subsection{Satisfacci\'on al cliente:}Se realiza seguimiento en cuanto al servicio al cliente en aspectos como atenci\'on, calidad en productos, aseo, etc.%
%
\section{Marco Referencial}
%
\textbf{CRM}: Costumer Relationship Manager, es una soluci\'on en sistemas de informaci\'on que automatiza e integra  los procesos de la organizaci\'on. Estos sistemas unen y sincronizan todas las operaciones de la organizaci\'on, tambi\'en permitiendo tener una conexi\'on entre la empresa con sus clientes y proveedores.%
\\%
\\%
Permite una interfaz con el usuario para ejecutar las transacciones de la empresa y bases de datos centralizada para almacenar toda la informaci\'on. Se consideran como la integraci\'on de los diferentes sistemas de informaci\'on en todas las \'areas de las empresas.%
\\%
\\%
Un ERP permite:%
\\%
\begin{itemize}
	\item Control,Gesti\'on y planeaci\'on de recursos.
	\item Planeaci\'on de productos.
	\item Gesti\'on de costos.
	\item Manejos de inventarios.
	\item F\'acil acceso a proveedores.
	\item Manejo de insumo.
\end{itemize}
%
El ERP consta de un flujo constante de informaci\'on entre las \'areas de la organizaci\'on las cuales tendr\'an un trabajo individual pero al mismo tiempo una coordinaci\'on e integraci\'on de los procesos que conforman el ERP.%
\\%
\\%
\textbf{Metodolog\'ia de desarrollo \'agil }: son m\'etodos de ingenier\'ia del software basados en el desarrollo iterativo e incremental, donde los requerimientos y soluciones evolucionan mediante la colaboraci\'on de grupos auto organizados y multidisciplinarios. Existen muchos m\'etodos de desarrollo \'agil; la mayor\'ia minimiza riesgos desarrollando software en lapsos cortos. El software desarrollado en una unidad de tiempo es llamado una iteraci\'on, la cual debe durar de una a cuatro semanas. Cada iteraci\'on del ciclo de vida incluye: planificaci\'on, an\'alisis de requerimientos, dise\~no, codificaci\'on, revisi\'on y documentaci\'on. Una iteraci\'on no debe agregar demasiada funcionalidad para justificar el lanzamiento del producto al mercado, pero la meta es tener una "demo" (sin errores) al final de cada iteraci\'on. Al final de cada iteraci\'on el equipo vuelve a evaluar las prioridades del proyecto.%
\section{ERP como soluci\'on adecuada}
%
La necesidad del negocio se puede resumir en los siguientes puntos:%
%
\begin{itemize}
\item Gesti\'on y soporte de los procesos para optimizaci\'on constante.
\item Control y seguimiento para permitir toma de desiciones.
\item Interconexi\'on e interoperabilidad entre las sucursales.
\end{itemize}
%
Teniendo en cuenta esto, la soluci\'on m\'as acertada es la implementaci\'on de un ERP, debido a las caracter\'isticas mencionadas anteriormente.%
\\%
\\%
Un sistema ERP permite gesti\'on unificada y descentralizada, permitiendo as\'i alcance a las sucursales sin traumatismos, permiti\'endoles ejercer como unidades independientes con un fin y una directriz en com\'un, con interacci\'on de forma asincr\'onica y en tiempo real, facilitando la gesti\'on y revisi\'on de los procesos.%
Con la automatizaci\'on de nuestra empresa por medio del ERP buscaremos:
%
\begin{itemize}
	\item Mejorar el flujo de procesos. 
	\item Obtener un mejor an\'alisis y respuestas.
	\item Disminuir costos de producci\'on y gesti\'on de inventario.
	\item Facilitar el servicio al cliente.
	\item Tener un control de insumos necesarios.
\end{itemize}
%
Despu\'es de un determinado tiempo de automatizar la organizaci\'on por medio del ERP, se obtendr\'an resultados y se puede justificar los costos de inversi\'on obteniendo como resultados, mejora de procesos y tiempos de entregas, reduciendo as\'i mismo, tiempo de espera  aumentando as\'i las ventas por un mejor servicio a los clientes y se puede lograr una mayor competitividad  en el mercado adem\'as, se logra un incremento notorio de precisi\'on de inventario y mejor gesti\'on de recursos.%
\\%
\\%
Al ser un medio tecnol\'ogico se busca tener el f\'acil acceso y ejecuci\'on de los usuarios as\'i mismo, los empleados tendr\'an una capacitaci\'on de c\'omo implementar esta nueva tecnolog\'ia ya que se le asignara responsabilidades m\'as grandes en la organizaci\'on. El ERP puede ser muy costoso al principio de su inversi\'on pero al manejar el f\'acil flujo de informaci\'on en los procesos de cada \'area de la empresa ser\'a una inversi\'on f\'acil de recuperar y de mejor proyecci\'on para la organizaci\'on.%
