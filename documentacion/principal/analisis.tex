\chapter{An\'alisis del sistema}
\section{Identificaci\'on de problemas y oportunidades b\'asicas}
En el estudio del sistema, se encontraron los siguientes problemas:%
\\
\begin{itemize}
\item El sistema debe ser flexible y parametrizable.%
\item El volumen del flujo de datos es muy grande.%
\item El registro de las ventas puede llegar a no estar acorde a la realidad.%
\item Los participantes pueden llegar a perder la tarjeta en la que se les realiza la consignaci\'on.%
\end{itemize}%
%
%\begin{figure}[htbp]
%	\centering
%		\includegraphics[width=0.60\textwidth]{images/analisisproblemas.png}
%	\caption{Matriz de problemas y oportunidades b\'asicas}
%	\label{fig:proboportunidades}
%\end{figure}%
%
\section{Evaluaci\'on del beneficio del proyecto}
Lo beneficios de el sistema de informaci\'on que se va a implementar nos permitir\'a consultar, observar y verificar la informaci\'on de la compa\~n\'ia en tiempo real y efectiva, de tal forma que estos procesos puedan ser flexibles y f\'acil de manejar para el usuario, esto nos lleva a que se Incrementara  el entendimiento del uso de las Tecnolog\'ias de Informaci\'on en la organizaci\'on y posteriormente  incrementar las ventas; para llevar esto a cabo, es necesario ser el primero en proporcionar informaci\'on a clientes potenciales sobre un producto en particular (Brinsamania) y mantener una relaci\'on con ellos a trav\'es de informaci\'on permanente.%
%
\subsection{?`Vale la pena trabajar en este proyecto?}
Hoy en d\'ia se esta tomando conciencia de la importancia dentro de las organizaciones de manejar los ambientes de negocio implementando un sistema de informaci\'on, lo fundamental para una compa\~n\'ia es un mercado globalizado a nivel competitivo, que tanto la estrategia de negocios y estrategias tecnol\'ogicas est\'en bajo un mismo esquema.%
\\%
\\%
Por lo anterior vale la pena trabajar en este proyecto ya que se ve reflejado la fusi\'on de estas dos estrategias, por una parte se esta trabajando los procesos del negocio y por el otro, se menejara un sistema que ayudar\'a a que todos estos procesos se realicen en forma automatizada, con el fin de alcanzar las metas y objetivos de la empresa.%
%
\subsection{?`Solucionar\'a los problemas?}
Unos de los principales inconvenientes que tiene la organizaci\'on es que el plan de incentivos que trabajan actualmente  se enmarca en un lapsos muy largos  (1 a\~no) lo que se busca es un sistema que implementara esta tarea pero en lapsos mas cortos (Aproximadamente cada dos meses), adem\'as de poder llevar la informaci\'on mas ordenada y precisa. Por medio de este sistema se lograr\'a un  incremento en  las ventas, ya que se va a poder manejar varias campa\~nas en lapsos mas cortos y atraer\'a a mas clientes, se ver\'a reflejado el ahorro de tiempo en la generaci\'on de reportes, se va a mantener actualizada y segura la informaci\'on de ventas y lo mas importante, la alta gerencia puede obtener la informaci\'on precisa y confiable en todo momento.%
%
\subsection{Agregar valor al proceso}
Se agrega valor en los siguientes aspectos:%
\begin{enumerate}
	\item Proporcionando informaci\'on a la alta gerencia sobre las habilidades de la organizaci\'on para llevar acabo las campa\~nas en tiempos tanto cortos como largos.
	\item Identificando inconvenientes que si resuelven y mejora la competencia y el incremento de las ventas, a trav\'es de la campa\~na (Brinsaman\'ia) y  el manejo de la tarjeta puntos en los clientes participantes.
	\item Identificando las oportunidades de mejora en el \'area de ventas.
\end{enumerate}%
%
\section{Dominio del problema}
El sistema de BrinsaMan\'ia es un sistema un tanto sencillo, desde el punto de vista de n\'umero de procesos que intervienen; siendo as\'i, un sistema con escasamente 2 actores, el participante o comprador mayorista y el usuario del sistema que puede ser el gerente de mercadeo, administrador de sistema o gerente de ventas.%
%
%\begin{figure}[htbp]
%	\centering
%		\includegraphics[width=0.60\textwidth]{images/contexto.png}
%	\caption{Diagrama de contexto del dominio del problema}
%	\label{fig:contexto}
%\end{figure}%
%
\section{An\'alisis de problemas y oportunidades}
Seg\'un la identificaci\'on anterior de problemas, el an\'alisis se interpret\'o de la siguiente forma:%
%
%\begin{figure}[htbp]
%	\centering
%		\includegraphics[width=0.60\textwidth]{images/problemasoportunidades.png}
%	\caption{Matriz de problemas y oportunidades b\'asicas}
%	\label{fig:anaproblemas}
%\end{figure}%
%
\section{An\'alisis de el proceso del negocio}
\subsection{Necesidad}
\begin{itemize}
	\item ?`Qui\'en usa el sistema?: Usuario administrador del sistema.
	\item ?`Qui\'en obtiene informaci\'on del sistema?: Mercadeo, alta gerencia, usuario administrador del sistema.
	\item ?`Qui\'en provee informaci\'on al sistema?: B.D.
	\item ?`Qui\'en soporta y mantiene el sistema?: Tecnolog\'ia.
	\item ?`D\'onde en la organizaci\'on se utiliza el sistema: Mercadeo.
\end{itemize}%
%
Se debe implementar un sistema que informaci\'on que desde el \'area de mercadeo  nos permita saber la manera mas adecuada para lanzar la campa\~na Brinsaman\'ia, y el tiempo de la misma, esto se lograr\'a  dependiendo del manejo de la informaci\'on que  por parte del usuario nos aporte y de esta manera, aplicarla a la campa\~na. Se debe aprovechar la informaci\'on que ya se encuentra almacenada en el sistema y  por medio de esta, nos permitir\'a recoger, consolidar y utilizar datos para que sean gestionados en la secci\'on comercial, con el fin de atraer mas clientes, incentiv\'andolos a que adquieran nuestros productos por medio de puntos que ser\'an acumulables en una tarjeta d\'ebito su equivalente en pesos.%
\\%
\\%
Los diferentes niveles de esta informaci\'on, nos permite conocer r\'apidamente desarrollos del mercado y tendencias econ\'omicas, con el fin de tomar las accciones apropiadas, dependiendo de lo que se quiera lograr. No s\'olo es posible recopilar y consolidar datos reales de gesti\'on comercial sino que adem\'as, puede crear sus propios datos planificados. Es necesario realizar la comparaci\'on entre los datos reales (incluyendo Nit Ingresados) y los datos  ingresados al sistema de manera err\'onea, esto nos  puede ayudar considerablemente a tomar mejores decisiones y medir la productividad de cada usuario con respecto a la campa\~na.%
%
\newpage%
\subsection{Diagramas de flujo}
%\begin{figure}[htbp]
%	\centering
%		\includegraphics[width=0.60\textwidth]{images/cuadritoese.png}
%	\caption{Proceso actual}
%	\label{fig:cuadritoese}
%\end{figure}%
%
\newpage%
\section{Identificaci\'on de requerimientos}
\subsection{Requerimientos no funcionales}
%\begin{figure}[htbp]
%	\centering
%		\includegraphics[width=1.00\textwidth]{images/nofuncionales.png}
%	\caption{Requerimientos no funcionales}
%	\label{fig:reqnofuncionales}
%\end{figure}%
%
\newpage%
\subsection{Requerimientos funcionales}
%\begin{figure}[htbp]
%	\centering
%		\includegraphics[width=1.00\textwidth]{images/funcionales.png}
%	\caption{Requerimientos funcionales}
%	\label{fig:reqfuncionales}
%\end{figure}%
%
\newpage%
\section{Priorizaci\'on de requerimientos del sistema}
%\begin{figure}[htbp]
%	\centering
%		\includegraphics[width=1.00\textwidth]{images/requerimientossitema.png}
%	\caption{Requerimientos del sistema}
%	\label{fig:reqsistema}
%\end{figure}%