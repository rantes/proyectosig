\chapter{Casos de uso}
\begin{figure}[htbp]
%centering es para centrar la imagen
	\centering
%aca es donde se incluye la imagen, se da el ancho(width), \textwidth significa que con repescto al tamano del
%texto y luego la ruta, relativa siempre es decir, a partir de donde se esta, como images esta ahi
%dentro, solo se usa desde images y ojala nada de espacios en el nombre de la imagen
		\includegraphics[width=0.60\textwidth]{images/casoinsumosgrafico.jpg}
%el caption es para el texto que aparece debajo de la imagen
	\caption{Caso de uso Comunity manager}
%label es para darle una referencia, por ejemplo si uno dice "como se puede ver en la imagen a1"
	\label{fig:Caso de uso Comunity manager}
\end{figure}%
\begin{figure}[htbp]
%centering es para centrar la imagen
	\centering
%aca es donde se incluye la imagen, se da el ancho(width), \textwidth significa que con repescto al tamano del
%texto y luego la ruta, relativa siempre es decir, a partir de donde se esta, como images esta ahi
%dentro, solo se usa desde images y ojala nada de espacios en el nombre de la imagen
		\includegraphics[width=0.60\textwidth]{images/casoinsumotabla.jpg}
%el caption es para el texto que aparece debajo de la imagen
	\caption{descripcion caso de uso de Comunity manager}
%label es para darle una referencia, por ejemplo si uno dice "como se puede ver en la imagen a1"
	\label{fig:descripcion caso de uso de Comunity manager}
\end{figure}%
\begin{figure}[htbp]
%centering es para centrar la imagen
	\centering
%aca es donde se incluye la imagen, se da el ancho(width), \textwidth significa que con repescto al tamano del
%texto y luego la ruta, relativa siempre es decir, a partir de donde se esta, como images esta ahi
%dentro, solo se usa desde images y ojala nada de espacios en el nombre de la imagen
		\includegraphics[width=0.60\textwidth]{images/casoproducciongrafico.jpg}
%el caption es para el texto que aparece debajo de la imagen
	\caption{Caso de uso de Cliente}
%label es para darle una referencia, por ejemplo si uno dice "como se puede ver en la imagen a1"
	\label{fig:Caso de uso de Cliente}
\end{figure}%
\begin{figure}[htbp]
%centering es para centrar la imagen
	\centering
%aca es donde se incluye la imagen, se da el ancho(width), \textwidth significa que con repescto al tamano del
%texto y luego la ruta, relativa siempre es decir, a partir de donde se esta, como images esta ahi
%dentro, solo se usa desde images y ojala nada de espacios en el nombre de la imagen
		\includegraphics[width=0.60\textwidth]{images/casoproducciontabla.jpg}
%el caption es para el texto que aparece debajo de la imagen
	\caption{descripcion caso de uso de cliente}
%label es para darle una referencia, por ejemplo si uno dice "como se puede ver en la imagen a1"
	\label{fig:descripcion caso de uso de cliente}
\end{figure}%
