\chapter{El Problema}\label{Problema}
\section{Descripci\'on del problema}
Al generarse un crecimiento de la productividad y ventas de la organizacion es necesario implementar un sistema de informaci\'on para optimizar y agilizar los procesos administrativos de la organizaci\'on.
Debido a la ausencia de dicho sistema no se logra percibir el impacto de las desiciones que se tomen en la organizaci\'on con respecto a los planes y programas de producci\'on y expansi\'on.%
\\%
\\%
Debido a que diariamente, el flujo transaccional es bastante fuerte y que no hay un buen sistema implementado que permita el soporte y gesti\'on de los procesos, dificilmente se puede gestionar campa\~nas o programas de mejoramiento u optimizaci\'on de procesos y muchas veces, p\'erdidas de dinero a causa de fallas en los procesos y la falta de registro en los flujos de los mismos.%
\\%
\\%
La panader\'ia Los Hornitos, ha crecido en forma gradual y han logrado expandirse y ahora cuentan con m\'as de 4 sucursales en la ciudad. Esto ha hecho m\'as dif\'icil la gesti\'on, ya que se requiere una centralizaci\'on de informaci\'on y una descentralizaci\'on de la administraci\'on de los procesos.%
%
\section{Formulaci\'on del problema}
En las diferentes alternativas de desarrollo de aplicaciones a la medida, ?`cual es la mejor manera de desarrollar una aplicaci\'on capaz de satisfacer a cabalidad las necesidades del negocio, soportando y gestionando todos los procesos?%
%
\section{Justificaci\'on}
La necesidad es demasiado espec\'ifica por ende, se hace prescindible el hecho de desarrollar una herramienta inform\'atica a la medida; en el mercado existen varias soluciones pero su funcionamiento es muy est\'andar y muchas veces, o se quedan cortas abarcando todas las necesidades o terminan ofreciendo mucho mas de lo que necesitan, convirtiendose en un dolor de cabeza por el exceso de actividades hasta innecesarias.%
\\%
\\%
Un sistema de informaci\'on hecho a la medida, permite que todos los requerimientos por parte del cliente, sean satisfechos y adem\'as le brinda tambi\'en tranquilidad y seguridad en los reportes y la gesti\'on de los procesos, dado que van a estar desarrollados a la medida, seg\'un sean requeridos y ajustados a la realidad de su negocio.%
\\%
\\%
La implementaci\'on de un nuevo Sistema De Gesti\'on Empresarial ERP se genera con la intenci\'on de optimizar procesos que se realizan dentro de la organizaci\'on haciendo que estas actividades tornen a ser mas factibles para los clientes. Al realizar esta nueva implementaci\'on, aportar\'a grandes beneficios a nivel competitivo, tambi\'en mejorado eficacia de los procesos ya que el Sistema de Gestion Empresarial permitir\'a optimizarlos.
%
\section{Alcance}
En seguida del estudio realizado y debido al conocimiento que se tiene de los diferentes procesos del las panaderias se implementar\'a un Sistema de Gesti\'on Empresarial(ERP), que ayudar\'a a facilitar el movimiento de la informaci\'on en inventarios, contabilidad, ventas y producci\'on para satisfacer sus necesidades de una gesti\'on m\'as dispuesta, mejorar la productividad y generar una mayor competitividad, en base al mejoramiento del sistema de informaci\'on.%