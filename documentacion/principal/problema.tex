\chapter{El Problema}\label{Problema}
\section{Descripci\'on del problema}
Al generarse un crecimiento de la productividad y ventas de la organizacion es necesario implementar un sistema de informaci\'on para optimizar y agilizar los procesos administrativos de la organizaci\'on.
Debido a la ausencia de dicho sistema no logra darse el cumplimiento a tiempo a las difrerentes peticiones de los usuarios de forma efectiva y r\'apida, ya que los empleados no tiene el sistema para poder generar mayor efectividad de atenci\'on y venta a los clientes. %
\\%
\\%
Debido a que diariamente se registran ventas y los clientes no son ordenados o mesurados con sus registros de compras, el procesos de cotejamiento de datos se hace muy complicado debido a que el sistema de gesti\'on de incentivos en compras se enlaza con otro sistema que es de incentivos a ventas y muchas veces, los datos de identificaci\'on no concuerdan.%

%
\section{Formulaci\'on del problema}
Cada a\~no las pol\'iticas de las campa\~nas de incentivos para clientes var\'ia, incluso se ven afectadas desde el punto de la planta de producci\'on; al incluir nuevas l\'ineas de productos o nuevos productos en las l\'ineas existentes.%
\\%
\\%
?`En las diferentes alternativas de desarrollo de aplicaciones a la medida, cual es la mejor manera de desarrollar una aplicaci\'on capaz de satisfacer a cabalidad las necesidades del plan que se dise\~ne y que brinde la informaci\'on oportuna y veraz para la gesti\'on de las campa\~nas como la de Brinsa Man\'ia?%
%
\section{Justificaci\'on}
La necesidad es demasiado espec\'ifica por ende, se hace prescindible el hecho de desarrollar una herramienta inform\'atica a la medida; en el mercado no existe algo parecido o est\'andar que pueda ser utilizado.%
\\%
\\%
Un sistema de informaci\'on hecho a la medida, permite que todos los requerimientos por parte del cliente, sean satisfechos y adem\'as le permite llevar una gesti\'on y seguimiento en tiempo real, adem\'as que puede tener pleno control sobre sus campa\~nas, brind\'andole tambi\'en tranquilidad en los reportes y gesti\'on de dineros, se hacen correctamente.%
\\%
\\%
La implementaci\'on de un nuevo Sistema De Gesti\'on Empresarial ERP se genera con la intenci\'on de optimizar procesos que se realizan dentro de la organizaci\'on haciendo que estas actividades tornen a ser mas factibles para los clientes.
\\%
\\%
Al realizar esta nueva implementaci\'on, aportara grandes beneficios a nivel competitivo, tambi\'en mejorado eficacia de los procesos ya que el sistema de gestion Empresarial se encargara de optimizarlos.
%
\section{Alcance}
En seguida del estudio realizado y debido al conocimiento que se tiene de los diferentes procesos del las panaderias se implementara un Sistema de Gesti\'on Empresarial, Enterprise Resource Planning ERP,que ayudara a facilitar el movimiento de la informaci\'on,inventarios, cantabilidad en concluci\'on satisface sus necesidades de una gesti\'on mas dispuesta, mejorar la productividad y generar una mayor competitividad, en base al mejoramiento del sistema de informaci\'on.%